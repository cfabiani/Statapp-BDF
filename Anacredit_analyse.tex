% Options for packages loaded elsewhere
\PassOptionsToPackage{unicode}{hyperref}
\PassOptionsToPackage{hyphens}{url}
%
\documentclass[
]{article}
\usepackage{amsmath,amssymb}
\usepackage{iftex}
\ifPDFTeX
  \usepackage[T1]{fontenc}
  \usepackage[utf8]{inputenc}
  \usepackage{textcomp} % provide euro and other symbols
\else % if luatex or xetex
  \usepackage{unicode-math} % this also loads fontspec
  \defaultfontfeatures{Scale=MatchLowercase}
  \defaultfontfeatures[\rmfamily]{Ligatures=TeX,Scale=1}
\fi
\usepackage{lmodern}
\ifPDFTeX\else
  % xetex/luatex font selection
\fi
% Use upquote if available, for straight quotes in verbatim environments
\IfFileExists{upquote.sty}{\usepackage{upquote}}{}
\IfFileExists{microtype.sty}{% use microtype if available
  \usepackage[]{microtype}
  \UseMicrotypeSet[protrusion]{basicmath} % disable protrusion for tt fonts
}{}
\makeatletter
\@ifundefined{KOMAClassName}{% if non-KOMA class
  \IfFileExists{parskip.sty}{%
    \usepackage{parskip}
  }{% else
    \setlength{\parindent}{0pt}
    \setlength{\parskip}{6pt plus 2pt minus 1pt}}
}{% if KOMA class
  \KOMAoptions{parskip=half}}
\makeatother
\usepackage{xcolor}
\usepackage[margin=1in]{geometry}
\usepackage{color}
\usepackage{fancyvrb}
\newcommand{\VerbBar}{|}
\newcommand{\VERB}{\Verb[commandchars=\\\{\}]}
\DefineVerbatimEnvironment{Highlighting}{Verbatim}{commandchars=\\\{\}}
% Add ',fontsize=\small' for more characters per line
\usepackage{framed}
\definecolor{shadecolor}{RGB}{248,248,248}
\newenvironment{Shaded}{\begin{snugshade}}{\end{snugshade}}
\newcommand{\AlertTok}[1]{\textcolor[rgb]{0.94,0.16,0.16}{#1}}
\newcommand{\AnnotationTok}[1]{\textcolor[rgb]{0.56,0.35,0.01}{\textbf{\textit{#1}}}}
\newcommand{\AttributeTok}[1]{\textcolor[rgb]{0.13,0.29,0.53}{#1}}
\newcommand{\BaseNTok}[1]{\textcolor[rgb]{0.00,0.00,0.81}{#1}}
\newcommand{\BuiltInTok}[1]{#1}
\newcommand{\CharTok}[1]{\textcolor[rgb]{0.31,0.60,0.02}{#1}}
\newcommand{\CommentTok}[1]{\textcolor[rgb]{0.56,0.35,0.01}{\textit{#1}}}
\newcommand{\CommentVarTok}[1]{\textcolor[rgb]{0.56,0.35,0.01}{\textbf{\textit{#1}}}}
\newcommand{\ConstantTok}[1]{\textcolor[rgb]{0.56,0.35,0.01}{#1}}
\newcommand{\ControlFlowTok}[1]{\textcolor[rgb]{0.13,0.29,0.53}{\textbf{#1}}}
\newcommand{\DataTypeTok}[1]{\textcolor[rgb]{0.13,0.29,0.53}{#1}}
\newcommand{\DecValTok}[1]{\textcolor[rgb]{0.00,0.00,0.81}{#1}}
\newcommand{\DocumentationTok}[1]{\textcolor[rgb]{0.56,0.35,0.01}{\textbf{\textit{#1}}}}
\newcommand{\ErrorTok}[1]{\textcolor[rgb]{0.64,0.00,0.00}{\textbf{#1}}}
\newcommand{\ExtensionTok}[1]{#1}
\newcommand{\FloatTok}[1]{\textcolor[rgb]{0.00,0.00,0.81}{#1}}
\newcommand{\FunctionTok}[1]{\textcolor[rgb]{0.13,0.29,0.53}{\textbf{#1}}}
\newcommand{\ImportTok}[1]{#1}
\newcommand{\InformationTok}[1]{\textcolor[rgb]{0.56,0.35,0.01}{\textbf{\textit{#1}}}}
\newcommand{\KeywordTok}[1]{\textcolor[rgb]{0.13,0.29,0.53}{\textbf{#1}}}
\newcommand{\NormalTok}[1]{#1}
\newcommand{\OperatorTok}[1]{\textcolor[rgb]{0.81,0.36,0.00}{\textbf{#1}}}
\newcommand{\OtherTok}[1]{\textcolor[rgb]{0.56,0.35,0.01}{#1}}
\newcommand{\PreprocessorTok}[1]{\textcolor[rgb]{0.56,0.35,0.01}{\textit{#1}}}
\newcommand{\RegionMarkerTok}[1]{#1}
\newcommand{\SpecialCharTok}[1]{\textcolor[rgb]{0.81,0.36,0.00}{\textbf{#1}}}
\newcommand{\SpecialStringTok}[1]{\textcolor[rgb]{0.31,0.60,0.02}{#1}}
\newcommand{\StringTok}[1]{\textcolor[rgb]{0.31,0.60,0.02}{#1}}
\newcommand{\VariableTok}[1]{\textcolor[rgb]{0.00,0.00,0.00}{#1}}
\newcommand{\VerbatimStringTok}[1]{\textcolor[rgb]{0.31,0.60,0.02}{#1}}
\newcommand{\WarningTok}[1]{\textcolor[rgb]{0.56,0.35,0.01}{\textbf{\textit{#1}}}}
\usepackage{graphicx}
\makeatletter
\def\maxwidth{\ifdim\Gin@nat@width>\linewidth\linewidth\else\Gin@nat@width\fi}
\def\maxheight{\ifdim\Gin@nat@height>\textheight\textheight\else\Gin@nat@height\fi}
\makeatother
% Scale images if necessary, so that they will not overflow the page
% margins by default, and it is still possible to overwrite the defaults
% using explicit options in \includegraphics[width, height, ...]{}
\setkeys{Gin}{width=\maxwidth,height=\maxheight,keepaspectratio}
% Set default figure placement to htbp
\makeatletter
\def\fps@figure{htbp}
\makeatother
\setlength{\emergencystretch}{3em} % prevent overfull lines
\providecommand{\tightlist}{%
  \setlength{\itemsep}{0pt}\setlength{\parskip}{0pt}}
\setcounter{secnumdepth}{-\maxdimen} % remove section numbering
\ifLuaTeX
  \usepackage{selnolig}  % disable illegal ligatures
\fi
\IfFileExists{bookmark.sty}{\usepackage{bookmark}}{\usepackage{hyperref}}
\IfFileExists{xurl.sty}{\usepackage{xurl}}{} % add URL line breaks if available
\urlstyle{same}
\hypersetup{
  pdftitle={Anacredit\_analyse},
  hidelinks,
  pdfcreator={LaTeX via pandoc}}

\title{Anacredit\_analyse}
\author{}
\date{\vspace{-2.5em}2024-03-12}

\begin{document}
\maketitle

\begin{Shaded}
\begin{Highlighting}[]
\FunctionTok{library}\NormalTok{(}\StringTok{"readxl"}\NormalTok{)}
\FunctionTok{library}\NormalTok{(}\StringTok{"here"}\NormalTok{)}
\end{Highlighting}
\end{Shaded}

\begin{verbatim}
## here() starts at C:/Users/cesar/Desktop/Repo Statapp/Statapp-BDF
\end{verbatim}

\begin{Shaded}
\begin{Highlighting}[]
\FunctionTok{library}\NormalTok{(}\StringTok{"FactoMineR"}\NormalTok{)}
\end{Highlighting}
\end{Shaded}

\begin{verbatim}
## Warning: package 'FactoMineR' was built under R version 4.3.2
\end{verbatim}

\begin{Shaded}
\begin{Highlighting}[]
\FunctionTok{library}\NormalTok{(}\StringTok{"factoextra"}\NormalTok{)}
\end{Highlighting}
\end{Shaded}

\begin{verbatim}
## Warning: package 'factoextra' was built under R version 4.3.2
\end{verbatim}

\begin{verbatim}
## Loading required package: ggplot2
\end{verbatim}

\begin{verbatim}
## Welcome! Want to learn more? See two factoextra-related books at https://goo.gl/ve3WBa
\end{verbatim}

\begin{Shaded}
\begin{Highlighting}[]
\FunctionTok{library}\NormalTok{(}\StringTok{"lubridate"}\NormalTok{)}
\end{Highlighting}
\end{Shaded}

\begin{verbatim}
## 
## Attaching package: 'lubridate'
\end{verbatim}

\begin{verbatim}
## The following objects are masked from 'package:base':
## 
##     date, intersect, setdiff, union
\end{verbatim}

\begin{Shaded}
\begin{Highlighting}[]
\FunctionTok{library}\NormalTok{(}\StringTok{"dplyr"}\NormalTok{)}
\end{Highlighting}
\end{Shaded}

\begin{verbatim}
## 
## Attaching package: 'dplyr'
\end{verbatim}

\begin{verbatim}
## The following objects are masked from 'package:stats':
## 
##     filter, lag
\end{verbatim}

\begin{verbatim}
## The following objects are masked from 'package:base':
## 
##     intersect, setdiff, setequal, union
\end{verbatim}

\begin{Shaded}
\begin{Highlighting}[]
\FunctionTok{library}\NormalTok{(}\StringTok{"tidyverse"}\NormalTok{)}
\end{Highlighting}
\end{Shaded}

\begin{verbatim}
## -- Attaching core tidyverse packages ------------------------ tidyverse 2.0.0 --
## v forcats 1.0.0     v stringr 1.5.0
## v purrr   1.0.1     v tibble  3.2.1
## v readr   2.1.4     v tidyr   1.3.0
\end{verbatim}

\begin{verbatim}
## -- Conflicts ------------------------------------------ tidyverse_conflicts() --
## x dplyr::filter() masks stats::filter()
## x dplyr::lag()    masks stats::lag()
## i Use the conflicted package (<http://conflicted.r-lib.org/>) to force all conflicts to become errors
\end{verbatim}

\begin{Shaded}
\begin{Highlighting}[]
\FunctionTok{library}\NormalTok{(}\StringTok{"readr"}\NormalTok{)}
\FunctionTok{library}\NormalTok{(}\StringTok{"psych"}\NormalTok{)}
\end{Highlighting}
\end{Shaded}

\begin{verbatim}
## Warning: package 'psych' was built under R version 4.3.2
\end{verbatim}

\begin{verbatim}
## 
## Attaching package: 'psych'
## 
## The following objects are masked from 'package:ggplot2':
## 
##     %+%, alpha
\end{verbatim}

\begin{Shaded}
\begin{Highlighting}[]
\FunctionTok{library}\NormalTok{(}\StringTok{"magrittr"}\NormalTok{)}
\end{Highlighting}
\end{Shaded}

\begin{verbatim}
## 
## Attaching package: 'magrittr'
## 
## The following object is masked from 'package:purrr':
## 
##     set_names
## 
## The following object is masked from 'package:tidyr':
## 
##     extract
\end{verbatim}

\begin{Shaded}
\begin{Highlighting}[]
\CommentTok{\# Ouverture de la base }
\NormalTok{anacredit}\OtherTok{=}\FunctionTok{read\_csv}\NormalTok{(}\StringTok{"D:/final\_df\_anonymous.csv"}\NormalTok{)}
\end{Highlighting}
\end{Shaded}

\begin{verbatim}
## Rows: 10337082 Columns: 35
## -- Column specification --------------------------------------------------------
## Delimiter: ","
## chr  (5): DBTR_ECNMC_ACTVTY_LE, DBTR_CNTRY, CRDTR_CNTRY, pseudonym_debtor, p...
## dbl (30): DT_RFRNC, DT_INCPTN, ORGNL_MTRTY, OTSTNDNG_NMNL_AMNT_CV, OFF_BLNC_...
## 
## i Use `spec()` to retrieve the full column specification for this data.
## i Specify the column types or set `show_col_types = FALSE` to quiet this message.
\end{verbatim}

\begin{Shaded}
\begin{Highlighting}[]
\CommentTok{\# Mise en format des Inception date}
\NormalTok{anacredit }\OtherTok{\textless{}{-}}\NormalTok{ anacredit }\SpecialCharTok{\%\textgreater{}\%}
  \FunctionTok{mutate}\NormalTok{(}\AttributeTok{DT\_INCPTN =} \FunctionTok{as.Date}\NormalTok{(}\FunctionTok{as.character}\NormalTok{(DT\_INCPTN), }\AttributeTok{format =} \StringTok{"\%Y\%m\%d"}\NormalTok{))}
\end{Highlighting}
\end{Shaded}

Statistiques descriptives

Première représentation des taux par jour

\begin{Shaded}
\begin{Highlighting}[]
\CommentTok{\#{-}{-}{-}{-}{-}Première représentation des taux par jour{-}{-}{-}{-}{-}}
\NormalTok{rate\_by\_day }\OtherTok{\textless{}{-}}\NormalTok{ anacredit }\SpecialCharTok{\%\textgreater{}\%}
  \FunctionTok{group\_by}\NormalTok{(DT\_INCPTN) }\SpecialCharTok{\%\textgreater{}\%}
  \FunctionTok{summarise}\NormalTok{(}\AttributeTok{mean\_rate =} \FunctionTok{mean}\NormalTok{(ANNLSD\_AGRD\_RT, }\AttributeTok{na.rm =} \ConstantTok{TRUE}\NormalTok{))}

\FunctionTok{ggplot}\NormalTok{(}\AttributeTok{data =}\NormalTok{ rate\_by\_day,  }\FunctionTok{aes}\NormalTok{(}\AttributeTok{x =}\NormalTok{ DT\_INCPTN, }\AttributeTok{y =}\NormalTok{ mean\_rate)) }\SpecialCharTok{+}
  \FunctionTok{geom\_line}\NormalTok{() }\SpecialCharTok{+}
  \FunctionTok{theme\_bw}\NormalTok{()}
\end{Highlighting}
\end{Shaded}

\includegraphics{Anacredit_analyse_files/figure-latex/unnamed-chunk-4-1.pdf}

\begin{Shaded}
\begin{Highlighting}[]
\CommentTok{\#{-}{-}{-}{-}{-}Tentative de débruitage{-}{-}{-}{-}{-}}
\CommentTok{\#Vérifier ces choix : on remplace les NA par des 0 et on ne sélectionne que les instruments avec une pd\textless{}=1}
\NormalTok{anacredit\_reduce }\OtherTok{\textless{}{-}}\NormalTok{ anacredit }\SpecialCharTok{\%\textgreater{}\%}
  \FunctionTok{mutate}\NormalTok{(}\AttributeTok{percentage\_protection =} \FunctionTok{ifelse}\NormalTok{(}\FunctionTok{is.na}\NormalTok{(percentage\_protection), }\DecValTok{0}\NormalTok{, percentage\_protection)) }\SpecialCharTok{\%\textgreater{}\%}
  \FunctionTok{filter}\NormalTok{(percentage\_protection }\SpecialCharTok{\textless{}=} \DecValTok{1} \SpecialCharTok{\&}\NormalTok{ percentage\_protection }\SpecialCharTok{!=} \ConstantTok{Inf}\NormalTok{) }\SpecialCharTok{\%\textgreater{}\%}
  \FunctionTok{mutate}\NormalTok{(}\AttributeTok{LTV\_INSTRMNT =} \FunctionTok{ifelse}\NormalTok{(}\FunctionTok{is.na}\NormalTok{(LTV\_INSTRMNT), }\DecValTok{0}\NormalTok{, LTV\_INSTRMNT)) }\SpecialCharTok{\%\textgreater{}\%}
  \FunctionTok{filter}\NormalTok{(}\SpecialCharTok{!}\FunctionTok{is.na}\NormalTok{(TYP\_INSTRMNT)) }\SpecialCharTok{\%\textgreater{}\%}
  \FunctionTok{filter}\NormalTok{(}\SpecialCharTok{!}\FunctionTok{is.na}\NormalTok{(ANNLSD\_AGRD\_RT))}

\CommentTok{\#On régresse}
\NormalTok{reg\_test }\OtherTok{\textless{}{-}} \FunctionTok{lm}\NormalTok{(ANNLSD\_AGRD\_RT }\SpecialCharTok{\textasciitilde{}}\NormalTok{ LTV\_INSTRMNT }\SpecialCharTok{+}\NormalTok{ TYP\_INSTRMNT }\SpecialCharTok{+}  
\NormalTok{                 percentage\_protection }\SpecialCharTok{+}\NormalTok{ OTSTNDNG\_NMNL\_AMNT\_CV }\SpecialCharTok{+}\NormalTok{ ORGNL\_MTRTY}
\NormalTok{                 , }\AttributeTok{data =}\NormalTok{ anacredit\_reduce)}
\FunctionTok{summary}\NormalTok{(reg\_test)}
\end{Highlighting}
\end{Shaded}

\begin{verbatim}
## 
## Call:
## lm(formula = ANNLSD_AGRD_RT ~ LTV_INSTRMNT + TYP_INSTRMNT + percentage_protection + 
##     OTSTNDNG_NMNL_AMNT_CV + ORGNL_MTRTY, data = anacredit_reduce)
## 
## Residuals:
##      Min       1Q   Median       3Q      Max 
## -100.037   -0.020   -0.002    0.017    8.870 
## 
## Coefficients:
##                         Estimate Std. Error t value Pr(>|t|)    
## (Intercept)            3.028e-02  3.846e-05 787.492  < 2e-16 ***
## LTV_INSTRMNT          -5.780e-11  1.616e-11  -3.577 0.000348 ***
## TYP_INSTRMNT           7.029e-06  5.074e-08 138.523  < 2e-16 ***
## percentage_protection  4.699e-03  8.407e-05  55.891  < 2e-16 ***
## OTSTNDNG_NMNL_AMNT_CV -2.832e-04  2.126e-05 -13.316  < 2e-16 ***
## ORGNL_MTRTY            5.547e-06  5.085e-08 109.090  < 2e-16 ***
## ---
## Signif. codes:  0 '***' 0.001 '**' 0.01 '*' 0.05 '.' 0.1 ' ' 1
## 
## Residual standard error: 0.06943 on 9735342 degrees of freedom
## Multiple R-squared:  0.004684,   Adjusted R-squared:  0.004683 
## F-statistic:  9163 on 5 and 9735342 DF,  p-value: < 2.2e-16
\end{verbatim}

\begin{Shaded}
\begin{Highlighting}[]
\CommentTok{\#Mais on trouve un R² très faible ...}
\end{Highlighting}
\end{Shaded}

\begin{Shaded}
\begin{Highlighting}[]
\CommentTok{\#On récupère les résidus}
\NormalTok{anacredit\_reduce}\SpecialCharTok{$}\NormalTok{ANNLSD\_AGRD\_RT\_clean }\OtherTok{=}\NormalTok{ reg\_test}\SpecialCharTok{$}\NormalTok{residuals}

\CommentTok{\#Représentation graphique}
\NormalTok{rate\_by\_day\_clean }\OtherTok{\textless{}{-}}\NormalTok{ anacredit\_reduce }\SpecialCharTok{\%\textgreater{}\%}
  \FunctionTok{group\_by}\NormalTok{(DT\_INCPTN) }\SpecialCharTok{\%\textgreater{}\%}
  \FunctionTok{summarise}\NormalTok{(}\AttributeTok{mean\_rate =} \FunctionTok{mean}\NormalTok{(ANNLSD\_AGRD\_RT\_clean, }\AttributeTok{na.rm =} \ConstantTok{TRUE}\NormalTok{))}


\FunctionTok{ggplot}\NormalTok{(}\AttributeTok{data =}\NormalTok{ rate\_by\_day\_clean,  }\FunctionTok{aes}\NormalTok{(}\AttributeTok{x =}\NormalTok{ DT\_INCPTN, }\AttributeTok{y =}\NormalTok{ mean\_rate)) }\SpecialCharTok{+}
  \FunctionTok{geom\_line}\NormalTok{() }\SpecialCharTok{+}
  \FunctionTok{theme\_bw}\NormalTok{()}
\end{Highlighting}
\end{Shaded}

\includegraphics{Anacredit_analyse_files/figure-latex/unnamed-chunk-6-1.pdf}

\begin{Shaded}
\begin{Highlighting}[]
\CommentTok{\# Pas d\textquotesingle{}amélioration particulière.{-}{-}\textgreater{} si regarder les ordonnées et un peu quand même}
\CommentTok{\# On peut tenter d\textquotesingle{}ajouter à la regression l\textquotesingle{}oustanding amount et la maturité mais il }
\CommentTok{\# faut clean ces variables d\textquotesingle{}abord}
\end{Highlighting}
\end{Shaded}


\end{document}
